%%%%%%%%%%%
%A template for completing the RCV analysis part of a report
%the rows of the table are printed with & separators by the .py file.
%You may want to have multiple tables for different settings (e.g. fewer candidates for one group).
%%%%%%%%%%%%

\documentclass{report}
\usepackage[dvipsnames]{xcolor}
\usepackage{colortbl}
\usepackage{xspace, parskip}
\usepackage[margin=1in]{geometry}

\newcommand{\POC}{POC\xspace} %could also be e.g. Black, Hispanic etc.

\begin{document}

We use four different models to estimate minority representation under ranked choice voting. All the models take a very simple input consisting of two values: 
\begin{itemize}
\item[(1)] the support from \POC voters for \POC -preferred candidates, and 
\item[(2)] the support from non-\POC voters for \POC -preferred candidates
\end{itemize}
The Placket-Luce-Dirichlet (PL) and Bradley-Terry-Dirichlet (BT) models rely on classical probabilistic models of ranking from the literature. The Alternating crossover (AC) and Cambridge sampler (CS) models rely on specific assumptions on how voters vote: the AC model assumes that crossover voters alternate between outgroup and ingroup candidates, while the CS model uses ballot data from a decade's worth of Cambridge MA city council races (which were ranked choice) to model voter behavior. We also consider five scenarios of how voters divide their support among non-\POC and \POC candidates. For the PL and BT models these scenarios are encoded in a parameter $\alpha$ displayed in the second row of the table below.

\begin{itemize}
\item Scenario A: unanimous order (all voters agree on who are the best candidates in each group).
\item Scenario B: \POC vary \POC (POC voters vary preferences among \POC -preferred candidates).
\item Scenario C: all vary order (no agreement on strongest candidates).
\item Scenario D: non-\POC vary non-\POC (non-\POC voters don’t agree on strongest candidates).
\item Scenario E: generic (all levels of agreement equally likely).
\end{itemize}


\begin{table}[h]
\centering
\begin{tabular}{ |c|c|c|c|c|c| }
\hline
\cellcolor{Gray}{\bf $\cdot$ seats}
 & Scenario A & Scenario B & Scenario C & Scenario D & average \\
\cline{2-6}
\cellcolor{Gray}{\bf $\cdot$ C / $\cdot$ candidates}
 & (.5, .5, .5, .5) & (2, .5, .5, .5) &  (2, 2, 2, 2) &  
 (.5, .5, 2, 2) & (1, 1, 1, 1) \\
\hline \hline
PL (Individual draws)  & $\cdot$  & $\cdot$ & $\cdot$ & $\cdot$ & $\cdot$ \\
\hline
BT (Paired comparisons) & $\cdot$  & $\cdot$ & $\cdot$ & $\cdot$ & $\cdot$ \\
\hline
Alternating crossover & $\cdot$  & $\cdot$ & $\cdot$ & $\cdot$ & $\cdot$ \\
\hline
Cambridge sampler & $\cdot$  & $\cdot$ & $\cdot$ & $\cdot$ & $\cdot$ \\
\hline
\end{tabular}
\end{table}


\end{document}
